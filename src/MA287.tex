\documentclass[12pt]{article}

%--------------------------------------------------------
% Basic Page Setup
%--------------------------------------------------------
\usepackage[margin=1in]{geometry} 
\usepackage{lmodern}
\usepackage[T1]{fontenc}
\usepackage[utf8]{inputenc}
\usepackage{microtype}  % Improves typography
\usepackage{pgfplots}
\pgfplotsset{compat=newest}  % Use the latest available features


% Slightly increase line spacing
\renewcommand{\baselinestretch}{1}

%--------------------------------------------------------
% Math Packages
%--------------------------------------------------------
\usepackage{amsmath,amssymb,amsthm}

%--------------------------------------------------------
% tcolorbox for Call-Out Blocks
%--------------------------------------------------------
\usepackage{tcolorbox}
\usepackage{xcolor}
\tcbuselibrary{theorems}

\definecolor{custom_green}{HTML}{a3be8c}
\definecolor{custom_red}{HTML}{bf616a}
\definecolor{custom_blue}{HTML}{5e81ac}
% Definition box
\newtcolorbox{definitionbox}{
  title=Definition,
  fonttitle=\bfseries,
  colback=blue!5!white,    % Light blue background
  colframe=blue!30!white,  % Darker blue frame
  coltitle=black,
  boxrule=1.5pt,
  arc=3pt,
  outer arc=3pt,
}

% Theorem box
\newtcolorbox{theorembox}{
  title=Theorem,
  fonttitle=\bfseries,
  colback=custom_green!30!white,
  colframe=custom_green,
  coltitle=black,
  boxrule=0.5pt,
  arc=4pt,
  outer arc=4pt,
}

% Note box
\newtcolorbox{examplebox}{
  title=Example,
  fonttitle=\bfseries,
  colback=custom_red!20!white,
  colframe=custom_red,
  coltitle=white,
  boxrule=0.5pt,
  arc=4pt,
  outer arc=4pt,
}


\newtcolorbox{proofbox}{
  title=Proof,
  fonttitle=\bfseries,
  colback=custom_blue!20!white,
  colframe=custom_blue,
  coltitle=white,
  boxrule=0.5pt,
  arc=4pt,
  outer arc=4pt,
}

\newtcolorbox{corollarybox}{
  title=Corollary,
  fonttitle=\bfseries,
  colback=custom_blue!20!white,
  colframe=custom_blue,
  coltitle=white,
  boxrule=0.5pt,
  arc=4pt,
  outer arc=4pt,
}

%--------------------------------------------------------
% Theorem Environments (optional alternative)
%--------------------------------------------------------
\theoremstyle{definition}
\newtheorem{definition}{Definition}[section]
\newtheorem{example}[definition]{Example}

\theoremstyle{plain}
\newtheorem{theorem}[definition]{Theorem}
% Slightly increase line spacing
\renewcommand{\baselinestretch}{1}

\title{
Complex Analysis\\[2ex]
Exams:\\
70\% Exam\\
30\% Continuous Assessment (Homework) \\
10\% Optional Project (Bonus)\\
}
\author{}     % Optional: Add author details if desired
\date{}       % Optional: Add date if desired
%--------------------------------------------------------
% Document
%--------------------------------------------------------
\begin{document}
\maketitle
\pagebreak

\tableofcontents
\pagebreak
\section{Week 1: Systems of Linear Equations}
\subsection{Intro to Systems of Linear Equations}
We call linear equations because each variable is raised to the first power. Products of variables, squares, square roots, etc., are not linear. 
A solution to a system of linear equations is an assignment of numerical values to each variables. Systems can have multiple solutions. 

\subsection{Augmented Matrices and Element row operations} 
\[
\begin{aligned}
x + 2y - z &= 5,\\
3x + y - 2z &= 9,\\
-x + 4y + 2z &= 0
\end{aligned}
\quad
\Longleftrightarrow
\quad 
\left(\begin{array}{@{} r r r | r @{}}
    1 & 2 & -1 & 5 \\
    3 & 1 & -2 & 9 \\
    -1 & 4 & 2 & 0
  \end{array}\right)
\]
To solve system of linear equations, we work with this augmented matrix, 
applying three types of operations to convert to a simpler form. These operations include:
\begin{enumerate}
    \item Adding scalar multiple of one row to another
    \item Multiplying all entries of a row by same non-zero scalar
    \item Swapping two rows. 
\end{enumerate}
\textbf{Why does this work?} Every ERO changes changes the system, but the new system has exactly the same solutions as the original. 

\begin{examplebox}

\[
\boxed{
\begin{aligned}
x_1 &+ 3x_2 + 5x_3 - 9x_4 &= 5,\\
3x_1 &- x_2 - 5x_3 + 13x_4 &= 5,\\
2x_1 &- 3x_2 - 8x_3 + 18x_4 &= 1.
\end{aligned}
}
\]
\[ \]
\[
\left(\!
\begin{array}{rrrr|r}
1 & 3 & 5 & -9 & 5\\
3 & -1 & -5 & 13 & 5\\
2 & -3 & -8 & 18 & 1\\
\end{array}
\!\right)
\quad\xrightarrow[R2 \to R2 - 3R1]{R3 \to R3 - 2R1}\quad
\left(\!
\begin{array}{rrrr|r}
1 & 3 & 5 & -9 & 5\\
0 & -10 & -20 & 40 & -10\\
0 & -9 & -18 & 36 & -9\\
\end{array}
\!\right)
\]
\[ \]
\[
\xrightarrow{R2 \times \left(-\tfrac{1}{10}\right)}
\left(\!
\begin{array}{rrrr|r}
1 & 3 & 5 & -9 & 5\\
0 & 1 & 2 & -4 & 1\\
0 & -9 & -18 & 36 & -9\\
\end{array}
\!\right)
\xrightarrow{R3 \to R3 + 9R2}
\left(\!
\begin{array}{rrrr|r}
1 & 3 & 5 & -9 & 5\\
0 & 1 & 2 & -4 & 1\\
0 & 0 & 0 & 0 & 0\\
\end{array}
\!\right)
\]
\[ \]
\[
\xrightarrow{R1 \to R1 - 3R2}
\left(\!
\begin{array}{rrrr|r}
1 & 0 & -1 & 3 & 2\\
0 & 1 & 2 & -4 & 1\\
0 & 0 & 0 & 0 & 0\\
\end{array}
\!\right)
\]

\end{examplebox}
\newpage
\subsection{How to read the solution from RREF}
\[
  \left[\!
\begin{array}{rrrr|r}
  1 & 0 & -1 & 3 & 2\\
  0 & 1 & 2 & -4 & 1\\
  0 & 0 & 0 & 0 & 0\\
  \end{array}
  \!\right]
\]
\[
\begin{array}{cccccccccccc}
  \text{Equation 1:}& \quad x_1 &+& 0 &-& x_3 &+& 3x_4 &=& 2\\
  \text{Equation 2:}& \quad 0 &+& x_2 &+& 2x_3 &-& 4x_4 &=& 1\\
  \text{Equation 3:}& \quad 0 &+& 0 &+& 0 &+& 0  &=& 0\\
\end{array}
\]

\noindent That is:
$$
\begin{aligned}
  &x_1 = 2 + x_3 - 3x_4 \\ 
  &x_2 = 1 - 2x_3 + 4x_4 \\
  &\text{Equation 3 has no content, so we can ignore it.} \\
\end{aligned}
$$

\noindent A solution of the system must satisfy all equations. We write $s$ and $t$ for the values of $x_3$ and $x_4$ in a solution of the system. The General Solution is written as:
\[(x_1, x_2, x_3, x_4) = (2 + s - 3t, 1 - 2s + 4t, s, t), \quad s, t \in \mathbb{R}\]
The general solution is a description of all the infinite solutions to the system. We can assign any values to $s$ and $t$ to get a solution. For example, if we set $s=0$ and $t=0$, we get the particular solution $(2,1,0,0)$.

\subsection{Vocabulary and Definitions}
\[
  \left[\!
  \begin{array}{rrrr|r}
  1 & 3 & 5 & -9 & 5\\
  3 & -1 & -5 & 13 & 5\\
  2 & -3 & -8 & 18 & 1\\
  \end{array}
  \!\right]
  \xrightarrow[\text{Elementary Row Operations}]{\text{Gauss-Jordan Elimination}}
  \left[\!
  \begin{array}{rrrr|r}
    1 & 3 & 5 & -9 & 5\\
    3 & -1 & -5 & 13 & 5\\
    2 & -3 & -8 & 18 & 1\\
  \end{array}
  \!\right]
\]
The second matrix is in reduced row echelon form (RREF). The RREF is a matrix where:
\begin{enumerate}
  \item Every row that is not all zeros has a leading 1.
  \item Every column thats containing a leady 1 has a 0 in every other entry.
  \item The leading 1 go left to right as we move down the rows
  \item Any rows that are all zero are at the bottom of the matrix.
\end{enumerate}
\textbf{Remark:}  A matrix is in row echelon form (REF) if it satisfies the first three conditions above. \\
\textbf{Example:} This matrix is in REF, not RREF.
\[
  \left[\!
  \begin{array}{rrrrr|r}
  1 & 2 & 3 & 0 & 4 & 5\\
  0 & 1 & 5 & 1 & -1 & 0\\
  0 & 0 & 0 & 0 & 1 & 20\\
  \end{array}
  \!\right]
\]
\newpage

\subsection{Leading Variables and Free Variables}
\[
\begin{align*}
  x_1 + 3x_2 + 5x_3 - 9x_4 &= 50\\
  3x_1 - x_2 - 5x_3 + 13x_4 &= 5\\
  2x_1 - 3x_2 - 8x_3 + 18x_4 &= 1.
\end{align*}
\rightarrow
\left[
\begin{array}{cccc|c}
  1 & 0 & -1 & 3 & 2 \\
  0 & 1 & 2 & -4 & 1 \\
  0 & 0 & 0 & 0 & 0
\end{array}
\right]
\]
\noindent Leading 1's in the RREF occur in the columns of the varables $x_1$ and $x_2$. The columns of $x_3$ and $x_4$ do not contain leading 1's. 

\begin{definitionbox}
  Any variables who {columns} in the RREF \textbf{do not contain leading 1's are called free variables.} \\
  Any variables who columns in the RREF \textbf{contain leading 1's are called leading variables.}
\end{definitionbox}

\subsection{How to write the solution}
\begin{enumerate}
  \item Give independant parameters to the free variables.
  \item Read from the RREF, how the correspodning values of the leading variables depend on s and t. $$(x_1, x_2, x_3, x_4) = (2+s - 3t, 1-2s+ 4t, s, t), \quad s,t \in \mathbb{R}$$
  \item \textbf{Note:} we can also write: $$(x_1, x_2, x_3, x_4) = (2,1,0,0) + s(1,-2,1,0) + t(-3, 4, 0, 1), \quad s,t \in \mathbb{R}$$
\end{enumerate}
\end{document}
