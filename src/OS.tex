\documentclass[10pt]{article}
\usepackage{lmodern}
\usepackage{amsmath}
\usepackage{amssymb}
\usepackage[most]{tcolorbox}
\usepackage{xcolor}

% Define custom purple color to match the image
\definecolor{definitionpurple}{RGB}{88, 44, 131}

% Create a clean definition box with left purple bar
\newtcolorbox{definition}[2][]{%
  enhanced,
  breakable,
  colback=white,
  colframe=definitionpurple!30!white,
  coltext=black,
  leftrule=5pt,
  rightrule=0.5pt,
  toprule=0.5pt,
  bottomrule=0.5pt,
  left=12pt,
  right=12pt,
  top=6pt,
  bottom=6pt,
  fonttitle=\large\sffamily\color{definitionpurple}\bfseries,
  title=#2,
overlay={
    \node[anchor=west, font=\sffamily\small\bfseries, 
          text=white, fill=definitionpurple, inner sep=3pt,
          minimum height=3mm] 
          at ([xshift=12pt]frame.north west) {\textsc Definition};
  },
attach boxed title to top left={xshift=-10pt},
title style={left=65pt},
  attach boxed title to top left,
  boxed title style={colback=white, colframe=white},
  #1
}

% Usage example
\begin{document}

\begin{definition}{Continuous Function}
    A function $f: X \rightarrow Y$ between topological spaces is said to be \textbf{continuous} if for every open set $V \subseteq Y$, the preimage $f^{-1}(V)$ is an open set in $X$.

    Equivalently, a function $f: \mathbb{R} \rightarrow \mathbb{R}$ is continuous at a point $c$ if for every $\varepsilon > 0$, there exists a $\delta > 0$ such that
    \begin{align}
        |x - c| < \delta \implies |f(x) - f(c)| < \varepsilon
    \end{align}
    for all $x$ in the domain of $f$.
\end{definition}

\end{document}