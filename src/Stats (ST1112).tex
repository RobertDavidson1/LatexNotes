\documentclass[a4paper, 10pt]{article}
% (1) Encoding, Fonts, and Layout
\usepackage[T1]{fontenc}
\usepackage{lmodern}
\usepackage[margin=1in]{geometry}


% (2) Common Packages
\usepackage{amsmath, amssymb, amsthm}
\usepackage{xcolor}
\usepackage{caption}
\usepackage{tikz}
\usepackage{pgfplots}
\pgfplotsset{compat=newest}
\usepackage{etoolbox}
\usepackage{tikz-3dplot}
\tdplotsetmaincoords{75}{120}
\usepackage[inline]{enumitem}
\usepackage{bookmark}
\usepackage{mathtools}
\usepackage{subcaption} % For subfigures
\usepackage[normalem]{ulem} % For better underline commands

% Micro-typography
\usepackage{microtype}

% Patching pgfplots warning
\makeatletter
\patchcmd{\pgfplots@applistXXpushback@smallbuf}{\pgfplots@error}{\pgfplots@warning}{}{}
\makeatother

% (3) tcolorbox and Theorem Libraries
\usepackage{tcolorbox}
\tcbuselibrary{theorems}

% (4) Define Colors
\definecolor{custom_green}{HTML}{a3be8c}
\definecolor{custom_red}{HTML}{dc322f}
\definecolor{custom_blue}{HTML}{268bd2}
\definecolor{custom_purple}{HTML}{b48ead}

\definecolor{base}{HTML}{eceff4}
\definecolor{gray1}{HTML}{e5e9f0}
\definecolor{gray2}{HTML}{d8dee9}
\definecolor{gray3}{HTML}{2e3440}
\pagecolor{base}

% (5) Custom tcolorbox Environments
\newtcolorbox{definitionbox}[1][]{
    title=\textbf{Definition} {#1},
    fonttitle=\bfseries\boldmath,
    arc=0mm,
    bottomtitle=0.5mm,
    boxrule=0mm,
    colbacktitle=gray2,
    colback=gray1,
    coltitle=gray3,
    coltext=gray3,
    left=2.5mm,
    leftrule=1mm,
    rightrule=1mm,
    right=3.5mm,
    toptitle=0.75mm,
    colframe=custom_red,
}

\newtcolorbox{proofbox}{
    title=\textbf{Proof},
    fonttitle=\bfseries\boldmath,
    arc=0mm,
    bottomtitle=0.5mm,
    boxrule=0mm,
    colbacktitle=gray2,
    colback=gray1,
    coltitle=gray3,
    left=2.5mm,
    leftrule=1mm,
    rightrule=1mm,
    right=3.5mm,
    toptitle=0.75mm,
    colframe=custom_blue,
    coltext=gray3,
}

\newtcolorbox{theorembox}[1][]{
    title=\textbf{Theorem} {#1},
    fonttitle=\bfseries\boldmath,
    arc=0mm,
    bottomtitle=0.5mm,
    boxrule=0mm,
    colbacktitle=gray2,
    colback=gray1,
    coltitle=gray3,
    left=2.5mm,
    leftrule=1mm,
    rightrule=1mm,
    right=3.5mm,
    toptitle=0.75mm,
    colframe=custom_green,
    coltext=gray3
}

\newtcolorbox{notebox}{
    title=\textbf{Note},
    fonttitle=\bfseries\boldmath,
    arc=0mm,
    bottomtitle=0.5mm,
    boxrule=0mm,
    colbacktitle=gray2,
    coltitle=gray3,
    left=2.5mm,
    leftrule=1mm,
    rightrule=1mm,
    right=3.5mm,
    toptitle=0.75mm,
    colframe=custom_blue,
    coltext=gray3
}

\newtcolorbox{examplebox}[1][]{
    title=\textbf{Example} {#1},
    fonttitle=\bfseries\boldmath,
    arc=0mm,
    bottomtitle=0.5mm,
    boxrule=0mm,
    colbacktitle=gray2,
    colback=gray1,
    coltitle=gray3,
    left=2.5mm,
    leftrule=1mm,
    rightrule=1mm,
    right=3.5mm,
    toptitle=0.75mm,
    colframe=gray3,
    fontupper=\footnotesize,
    coltext=gray3
}

% (6) Theorem Environments
\theoremstyle{definition}
\newtheorem{definition}{Definition}[section]
\newtheorem{example}[definition]{Example}

\theoremstyle{plain}
\newtheorem{theorem}[definition]{Theorem}

% (7) Hyperlinks
\usepackage{hyperref}
\hypersetup{
    colorlinks=true,    % Use colored text for links
    linkcolor=custom_red,      % Set link text color to red
    pdfborder={0 0 0}   % Remove the default box around links
}


% macros.tex
\newcommand{\intinf}{\int_0^{\infty}} % Integral from 0 to infinity
\newcommand{\diff}[2]{\frac{d#1}{d#2}} % Derivative


\usepackage[svgnames]{xcolor}
\usepackage{listings}

\lstset{language=R,
    basicstyle=\small\ttfamily,
    stringstyle=\color{DarkGreen},
    otherkeywords={0,1,2,3,4,5,6,7,8,9},
    morekeywords={TRUE,FALSE},
    deletekeywords={data,frame,length,as,character},
    keywordstyle=\color{blue},
    commentstyle=\color{DarkGreen},
}

\title{
Robert Davidson \\
\textbf{ST1112: Statistics}
}
\author{
70\% Exam\\
30\% Continuous Assessment (3 parts)
}
\date{}       % Optional: Add date if desired
%--------------------------------------------------------
% Document
%--------------------------------------------------------
\begin{document}
\maketitle
\pagebreak



\tableofcontents
\pagebreak
\section{Descriptive Statisitcs}
\subsection{Sampling the mean}
In \textbf{probability} we consider the underlying process which has some randomness or uncertainity, and we try to figure out what happens \\[2ex]
In \textbf{statistics} we consider the data that we have, and we try to figure out what the underlying process is. The basic aim to infer the population from the sample.
\begin{examplebox}[Consider a jar of red and green jelly beans]
    A probabilist starts by knowing the proportion of red and green jelly beans in the jar, and then tries to figure out the probability of drawing a red jelly bean.\\
    A statistician starts by drawing a sample of jelly beans from the jar, and then tries to figure out the proportion of red and green jelly beans in the jar.
\end{examplebox}

\begin{definitionbox}[ : Central Limit Theorem]
    Sample means follow a normal distrubution, centered on the popular mean, with a standard deviation equal to population standard deviation divided by the square root of the sample size.
    $$\bar{X} \sim N \left(\mu, \frac{\sigma}{\sqrt{n}}\right)$$
\end{definitionbox}

\begin{definitionbox}[ : Standard Error]
    The standard error is the variability in the sampling distrubution. \\
    The standard error describes the typical difference between the sample measurement and the population parameter.
    $$SE = \frac{\sigma}{\sqrt{n}}$$
\end{definitionbox}
\begin{definitionbox}[ : Estimate $\sigma$]
    Often the value of the population standard deviation is unknown, and hence the standard error of the mean is unknown. \\
    We can estimate the value of the standard error using the sample standard deviation $(s)$ as an unbiased estimator of the population standard deviation $(\sigma)$.
    $$\sigma_{\bar{X}} = \frac{s}{\sqrt{n}}$$
\end{definitionbox}

\pagebreak

\section{Interential Statistics - Interval Estimation}
\subsection{Confidence Intervals for a mean}
\begin{examplebox}[The student newspaper wants to know how many students are exercising per week on average]
    \begin{itemize}
        \item Take a \textbf{sample from this population}
        \item Esimate the \textbf{population paramater} using the \textbf{sample statistic}
    \end{itemize}

    \begin{lstlisting}
        st1112_data %>% 
            select(exercise) %>% 
            summarise(n = n(),
                mean = mean(exercise,na.rm=TRUE),
                sd = sd(exercise,na.rm=TRUE))
        
        ##    n mean   sd
        ## 1 54 6.19 4.11

        st1112_data %>% 
            ggplot(aes(x = exercise)) + 
            labs(x = "Weekly exercise (hours)") +
            geom_density(colour = "blue", 
                         fill="blue",alpha=0.5)+theme_bw()
    \end{lstlisting}
    \begin{center}
        \includegraphics[scale=0.5]{images/image.png}
    \end{center}

    But a new survery on another 54 students would lead to a different estimate, so which should we report back to the newspaper?
    If we sample data from the population, there is uncertainity in our estimate of the population mean.
    The standard error of the mean is a measure of this uncertainity. In our example, the standard error of the mean is:
    $$SE = \frac{4.11}{\sqrt{54}} = 0.56$$
    We use the Central Limit  Theorem to provide a range of values that will capture $95\%$ of the sample means
\end{examplebox}

\pagebreak

\begin{definitionbox}[Confidence Interval for $n > 30$]
    For a large sample size, $n > 30$, a Confidence Interval for the population mean is given by:
    $$\bar{X} \pm Z_{\frac{\alpha}{2}} \frac{s}{\sqrt{n}}$$
\end{definitionbox}
For a $95\%$ Confidence Interval, $\alpha = 0.05$, and $Z_{\frac{\alpha}{2}} = Z_{0.25}$.

\end{document}
