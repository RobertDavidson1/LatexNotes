\documentclass[4pt]{article}

%--------------------------------------------------------
% Basic Page Setup
%--------------------------------------------------------
\usepackage[margin=1in]{geometry}
\usepackage{lmodern}
\usepackage[T1]{fontenc}
\usepackage[utf8]{inputenc}
\usepackage{microtype}  % Improves typography
\usepackage{pgfplots}
\pgfplotsset{compat=newest}  % Use the latest available features

% Slightly increase line spacing
\renewcommand{\baselinestretch}{1}

%--------------------------------------------------------
% Math Packages
%--------------------------------------------------------
\usepackage{amsmath,amssymb,amsthm}

%--------------------------------------------------------
% tcolorbox for Call-Out Blocks
%--------------------------------------------------------
\usepackage{tcolorbox}
\usepackage{xcolor}
\tcbuselibrary{theorems}

\definecolor{custom_green}{HTML}{a3be8c}
\definecolor{custom_red}{HTML}{bf616a}
\definecolor{custom_blue}{HTML}{5e81ac}

% Definition box
\newtcolorbox{definitionbox}{
  title=Definition,
  fonttitle=\bfseries,
  colback=blue!5!white,    % Light blue background
  colframe=blue!30!white,  % Darker blue frame
  coltitle=black,
  boxrule=1.5pt,
  arc=3pt,
  outer arc=3pt,
}

% Theorem box
\newtcolorbox{theorembox}{
  title=Theorem,
  fonttitle=\bfseries,
  colback=custom_green!30!white,
  colframe=custom_green,
  coltitle=black,
  boxrule=0.5pt,
  arc=4pt,
  outer arc=4pt,
}

% Example box
\newtcolorbox{examplebox}{
  title=Example,
  fonttitle=\bfseries,
  colback=custom_red!20!white,
  colframe=custom_red,
  coltitle=white,
  boxrule=0.5pt,
  arc=4pt,
  outer arc=4pt,
}

% Proof box
\newtcolorbox{proofbox}{
  title=Proof,
  fonttitle=\bfseries,
  colback=custom_blue!20!white,
  colframe=custom_blue,
  coltitle=white,
  boxrule=0.5pt,
  arc=4pt,
  outer arc=4pt,
}

% Corollary box
\newtcolorbox{corollarybox}{
  title=Corollary,
  fonttitle=\bfseries,
  colback=custom_blue!20!white,
  colframe=custom_blue,
  coltitle=white,
  boxrule=0.5pt,
  arc=4pt,
  outer arc=4pt,
}

%--------------------------------------------------------
% Theorem Environments (optional alternative)
%--------------------------------------------------------
\theoremstyle{definition}
\newtheorem{definition}{Definition}[section]
\newtheorem{example}[definition]{Example}

\theoremstyle{plain}
\newtheorem{theorem}[definition]{Theorem}

%--------------------------------------------------------
% Title, Author, Date
%--------------------------------------------------------
\title{
Linear Algebra\\[2ex]
Exams:\\
70\% Exam\\
30\% Continuous Assessment (Homework) \\
10\% Optional Project (Bonus)\\
}
\author{}     % Optional: Add author details if desired
\date{}       % Optional: Add date if desired

%--------------------------------------------------------
% Document
%--------------------------------------------------------
\begin{document}

\maketitle
\pagebreak

\tableofcontents
\pagebreak

\section{Week 1: Systems of Linear Equations}

\subsection{Intro to Systems of Linear Equations}
They are called linear equations because each variable is raised to the first power. Products of variables, squares, square roots, etc., are not linear.
A solution to a system of linear equations is an assignment of numerical values to each variable. Systems can have multiple solutions.

\subsection{Augmented Matrices and Elementary Row Operations}

\begin{align}
  x + 2y - z   & = 5, \\
  3x + y - 2z  & = 9, \\
  -x + 4y + 2z & = 0
\end{align}
\[
  \Longleftrightarrow
  \quad
  \left(\begin{array}{@{} r r r | r @{}}
      1  & 2 & -1 & 5 \\
      3  & 1 & -2 & 9 \\
      -1 & 4 & 2  & 0
    \end{array}\right)
\]
To solve a system of linear equations, we work with this augmented matrix,
applying three types of operations to convert it to a simpler form. These operations include:
\begin{enumerate}
  \item Adding a scalar multiple of one row to another
  \item Multiplying all entries of a row by the same non-zero scalar
  \item Swapping two rows
\end{enumerate}
\textbf{Why does this work?} Every elementary row operation (ERO) changes the system, but the new system has exactly the same solutions as the original.

\begin{examplebox}
  \[
    \boxed{
      \begin{aligned}
        x_1  & + 3x_2 + 5x_3 - 9x_4  & = 5, \\
        3x_1 & - x_2 - 5x_3 + 13x_4  & = 5, \\
        2x_1 & - 3x_2 - 8x_3 + 18x_4 & = 1.
      \end{aligned}
    }
  \]
  \[
    \left(\!
    \begin{array}{rrrr|r}
        1 & 3  & 5  & -9 & 5 \\
        3 & -1 & -5 & 13 & 5 \\
        2 & -3 & -8 & 18 & 1 \\
      \end{array}
    \!\right)
    \quad \xrightarrow[R2 \to R2 - 3R1]{R3 \to R3 - 2R1} \quad
    \left(\!
    \begin{array}{rrrr|r}
        1 & 3   & 5   & -9 & 5   \\
        0 & -10 & -20 & 40 & -10 \\
        0 & -9  & -18 & 36 & -9  \\
      \end{array}
    \!\right)
  \]
  \[
    \xrightarrow{R2 \times \left(-\tfrac{1}{10}\right)}
    \left(\!
    \begin{array}{rrrr|r}
        1 & 3  & 5   & -9 & 5  \\
        0 & 1  & 2   & -4 & 1  \\
        0 & -9 & -18 & 36 & -9 \\
      \end{array}
    \!\right)
    \xrightarrow{R3 \to R3 + 9R2}
    \left(\!
    \begin{array}{rrrr|r}
        1 & 3 & 5 & -9 & 5 \\
        0 & 1 & 2 & -4 & 1 \\
        0 & 0 & 0 & 0  & 0 \\
      \end{array}
    \!\right)
  \]
  \[
    \xrightarrow{R1 \to R1 - 3R2}
    \left(\!
    \begin{array}{rrrr|r}
        1 & 0 & -1 & 3  & 2 \\
        0 & 1 & 2  & -4 & 1 \\
        0 & 0 & 0  & 0  & 0 \\
      \end{array}
    \!\right)
  \]
\end{examplebox}

\newpage

\subsection{How to Read the Solution from RREF}
\[
  \left[\!
    \begin{array}{rrrr|r}
      1 & 0 & -1 & 3  & 2 \\
      0 & 1 & 2  & -4 & 1 \\
      0 & 0 & 0  & 0  & 0 \\
    \end{array}
    \!\right]
\]
\[
  \begin{array}{r@{\;}l}
    \text{Equation 1:} & \quad x_1 - x_3 + 3x_4 = 2,  \\
    \text{Equation 2:} & \quad x_2 + 2x_3 - 4x_4 = 1, \\
    \text{Equation 3:} & \quad 0 = 0.
  \end{array}
\]

Thus:
\[
  x_1 = 2 + x_3 - 3x_4,
  \quad
  x_2 = 1 - 2x_3 + 4x_4.
\]
We see \(x_3\) and \(x_4\) can be any real values. Let \(s = x_3\) and \(t = x_4\). Then
\[
  (x_1, x_2, x_3, x_4)
  =
  (2 + s - 3t,\; 1 - 2s + 4t,\; s,\; t),
  \quad
  s, t \in \mathbb{R}.
\]
There are infinitely many solutions. For instance, \((2,1,0,0)\) is one particular solution.

\subsection{Vocabulary and Definitions}
\[
  \left[\!
    \begin{array}{rrrr|r}
      1 & 3  & 5  & -9 & 5 \\
      3 & -1 & -5 & 13 & 5 \\
      2 & -3 & -8 & 18 & 1 \\
    \end{array}
    \!\right]
  \;\xrightarrow[\text{ERO}]{\text{Gauss-Jordan Elimination}}\;
  \left[\!
    \begin{array}{rrrr|r}
      1 & 0 & -1 & 3  & 2 \\
      0 & 1 & 2  & -4 & 1 \\
      0 & 0 & 0  & 0  & 0 \\
    \end{array}
    \!\right]
\]
The second matrix is in \emph{reduced row echelon form (RREF)}. A matrix is in RREF if:
\begin{enumerate}
  \item Every nonzero row has a leading 1 (pivot).
  \item Each pivot is the only nonzero entry in its column.
  \item The pivots move strictly to the right as you go down the rows.
  \item Any all-zero rows appear at the bottom.
\end{enumerate}
\textbf{Remark:} A matrix is in \emph{row echelon form (REF)} if it merely satisfies the “strictly right pivots” and “zeros below pivots,” plus any zero rows at the bottom. RREF is the fully reduced form.

\[
  \textbf{Example (REF but not RREF):}\quad
  \left[\!
    \begin{array}{rrrrr|r}
      1 & 2 & 3 & 0 & 4  & 5  \\
      0 & 1 & 5 & 1 & -1 & 0  \\
      0 & 0 & 0 & 0 & 1  & 20 \\
    \end{array}
    \!\right]
\]

\subsection{Leading Variables and Free Variables}
\[
  \begin{aligned}
    x_1 + 3x_2 + 5x_3 - 9x_4   & = 50, \\
    3x_1 - x_2 - 5x_3 + 13x_4  & = 5,  \\
    2x_1 - 3x_2 - 8x_3 + 18x_4 & = 1
  \end{aligned}
  \;\;\longrightarrow\;\;
  \left[
    \begin{array}{cccc|c}
      1 & 0 & -1 & 3  & 2 \\
      0 & 1 & 2  & -4 & 1 \\
      0 & 0 & 0  & 0  & 0
    \end{array}
    \right].
\]
Leading 1's in the RREF appear in columns 1 and 2 (so \(x_1\) and \(x_2\) are leading variables). Columns 3 and 4 have no pivot (so \(x_3\) and \(x_4\) are free).

\begin{definitionbox}
  A variable is called a \textbf{free variable} if its column in the RREF does \emph{not} contain a leading 1. \\
  A variable is called a \textbf{leading variable} if its column in the RREF \emph{does} contain a leading 1.
\end{definitionbox}

\subsection{How to Write the Solution}
\begin{enumerate}
  \item Assign independent parameters to each free variable.
  \item Read off the dependencies for the leading variables from the RREF.
  \item Optionally, write the solution in vector form:
        \[
          (x_1, x_2, x_3, x_4)
          \;=\;
          (2,1,0,0)
          \;+\;
          s\,(1,-2,1,0)
          \;+\;
          t\,(-3,4,0,1).
        \]
\end{enumerate}

\section{Inconsistent Systems}
\subsection{Gauss Jordan Elimination Algorithm}
\begin{enumerate}
  \item Get a 1 in the upper left corner (unless Column 1 is all zeros).
  \item Use this 1 to clear out the rest of Column 1 by adding/subtracting multiples of Row 1 to subsequent rows. (Do not touch Row 1 again until step 5 )
  \item Move to Column 2. Without disturbing the pattern of zeros in Column 1, get a 1 in the second row. Use this 1 to clear out the rest of Column 2.
  \item Proceed through the columns in this manner. (Gives row echelon form.)
  \item Use the leading 1s to clear out the columns above them, working left to right
\end{enumerate}

\begin{examplebox}
  \[
    \begin{aligned}
       & 3x & + & 2y & - & 5z & = & 4 \\
       & x  & + & y  & - & 2z & = & 1 \\
       & 5x & + & 3y & - & 8z & = & 6
    \end{aligned}
    \;\;\longrightarrow\;\;
    \left[
      \begin{array}{ccc|c}
        3 & 2 & -5 & 4 \\
        1 & 1 & -2 & 1 \\
        5 & 3 & -8 & 6
      \end{array}
      \right].
  \]
  During Gausian Elimination of the system below, a row is encountered that has all zero entries, excpet the last, i.e. \(0x + 0y + 0z = 1\). This is a contradiction, and the system is inconsistent. \\
  A system is inconsistent if its equations can be be simultaneously satisfied, i.e. there is no solution to:
  \[
    x+y = 1 \quad \text{and} \quad x+y = 2
  \]
\end{examplebox}

\subsection{Possible outcomes to solving a linear system}
It is not possible for a linear system to have exactly two solutions. The possible outcomes are:
\begin{enumerate}
  \item \textbf{Unique Solution:} The system has exactly one solution.
  \item \textbf{Infinite Solutions:} The system has infinitely many solutions.
  \item \textbf{No Solution:} The system is inconsistent.
\end{enumerate}

\subsection{Pitfalls of Gaussian Elimination}
Algorithms,, based on Gauss Jordan Elimination, for solving systems of linear equations are not foolproof. They can fail in the following ways:
\begin{enumerate}
  \item \textbf{Numerical Instability:} Round-off errors can accumulate and lead to incorrect results.
  \item \textbf{Computational Complexity:} The number of operations requiredis bounded above by a cubic expression in n. This becomes impractical for large systems, where iterative methods are used.
\end{enumerate}
\pagebreak
\end{document}
