\documentclass[a4paper, 9pt]{extarticle}
% (1) Encoding, Fonts, and Layout
\usepackage[T1]{fontenc}
\usepackage{lmodern}
\usepackage[margin=1in]{geometry}


% (2) Common Packages
\usepackage{amsmath, amssymb, amsthm}
\usepackage{xcolor}
\usepackage{caption}
\usepackage{tikz}
\usepackage{pgfplots}
\pgfplotsset{compat=newest}
\usepackage{etoolbox}
\usepackage{tikz-3dplot}
\tdplotsetmaincoords{75}{120}
\usepackage[inline]{enumitem}
\usepackage{bookmark}
\usepackage{mathtools}
\usepackage{subcaption} % For subfigures
\usepackage[normalem]{ulem} % For better underline commands

% Micro-typography
\usepackage{microtype}

% Patching pgfplots warning
\makeatletter
\patchcmd{\pgfplots@applistXXpushback@smallbuf}{\pgfplots@error}{\pgfplots@warning}{}{}
\makeatother

% (3) tcolorbox and Theorem Libraries
\usepackage{tcolorbox}
\tcbuselibrary{theorems}

% (4) Define Colors
\definecolor{custom_green}{HTML}{a3be8c}
\definecolor{custom_red}{HTML}{dc322f}
\definecolor{custom_blue}{HTML}{268bd2}
\definecolor{custom_purple}{HTML}{b48ead}

\definecolor{base}{HTML}{eceff4}
\definecolor{gray1}{HTML}{e5e9f0}
\definecolor{gray2}{HTML}{d8dee9}
\definecolor{gray3}{HTML}{2e3440}
\pagecolor{base}

% (5) Custom tcolorbox Environments
\newtcolorbox{definitionbox}[1][]{
    title=\textbf{Definition} {#1},
    fonttitle=\bfseries\boldmath,
    arc=0mm,
    bottomtitle=0.5mm,
    boxrule=0mm,
    colbacktitle=gray2,
    colback=gray1,
    coltitle=gray3,
    coltext=gray3,
    left=2.5mm,
    leftrule=1mm,
    rightrule=1mm,
    right=3.5mm,
    toptitle=0.75mm,
    colframe=custom_red,
}

\newtcolorbox{proofbox}{
    title=\textbf{Proof},
    fonttitle=\bfseries\boldmath,
    arc=0mm,
    bottomtitle=0.5mm,
    boxrule=0mm,
    colbacktitle=gray2,
    colback=gray1,
    coltitle=gray3,
    left=2.5mm,
    leftrule=1mm,
    rightrule=1mm,
    right=3.5mm,
    toptitle=0.75mm,
    colframe=custom_blue,
    coltext=gray3,
}

\newtcolorbox{theorembox}[1][]{
    title=\textbf{Theorem} {#1},
    fonttitle=\bfseries\boldmath,
    arc=0mm,
    bottomtitle=0.5mm,
    boxrule=0mm,
    colbacktitle=gray2,
    colback=gray1,
    coltitle=gray3,
    left=2.5mm,
    leftrule=1mm,
    rightrule=1mm,
    right=3.5mm,
    toptitle=0.75mm,
    colframe=custom_green,
    coltext=gray3
}

\newtcolorbox{notebox}{
    title=\textbf{Note},
    fonttitle=\bfseries\boldmath,
    arc=0mm,
    bottomtitle=0.5mm,
    boxrule=0mm,
    colbacktitle=gray2,
    coltitle=gray3,
    left=2.5mm,
    leftrule=1mm,
    rightrule=1mm,
    right=3.5mm,
    toptitle=0.75mm,
    colframe=custom_blue,
    coltext=gray3
}

\newtcolorbox{examplebox}[1][]{
    title=\textbf{Example} {#1},
    fonttitle=\bfseries\boldmath,
    arc=0mm,
    bottomtitle=0.5mm,
    boxrule=0mm,
    colbacktitle=gray2,
    colback=gray1,
    coltitle=gray3,
    left=2.5mm,
    leftrule=1mm,
    rightrule=1mm,
    right=3.5mm,
    toptitle=0.75mm,
    colframe=gray3,
    fontupper=\footnotesize,
    coltext=gray3
}

% (6) Theorem Environments
\theoremstyle{definition}
\newtheorem{definition}{Definition}[section]
\newtheorem{example}[definition]{Example}

\theoremstyle{plain}
\newtheorem{theorem}[definition]{Theorem}

% (7) Hyperlinks
\usepackage{hyperref}
\hypersetup{
    colorlinks=true,    % Use colored text for links
    linkcolor=custom_red,      % Set link text color to red
    pdfborder={0 0 0}   % Remove the default box around links
}

% macros.tex
\newcommand{\intinf}{\int_0^{\infty}} % Integral from 0 to infinity
\newcommand{\diff}[2]{\frac{d#1}{d#2}} % Derivative


\title{
\textbf{MA283: Linear Algebra} \\ 
}

\author{
  70\% Exam\\
30\% Continuous Assessment (Homework) \\
10\% Optional Project (Bonus)\\ [2ex]
Robert Davidson
}     
\date{}       % Optional: Add date if desired

\begin{document}

\maketitle
\pagebreak

\tableofcontents
\pagebreak
\section{Systems of linear equations}
\subsection{Linear equations and Solution Sets}
A linear equation in the variables $x$ and
$y$ is an equation of the form
\begin{equation*}
  2x + y = 3
\end{equation*}
If we replace $x$ and $y$ with some numbers, the statement \textbf{becomes true or false}.

\begin{definitionbox}{Solution to a linear equation}{}
  A pair, $(x_0, y_0) \in \mathbb{R}$, is a solution to an linear equation if setting $x = x_0$ and $y = y_0$ \textbf{makes the equation true.}
\end{definitionbox}

\begin{definitionbox}{Solution set}{}
  The \textbf{solution set} is the set of all solutions to a linear equation.
  $$a_1X_1 + a_2X_2 + \ldots + a_nX_n = b \quad \text{where} \; a_i, b \in \mathbb{R}$$
  is an \textbf{affine hyperplane} in $\mathbb{R}^n$; geometrically resembles a copy of $\mathbb{R}^{n-1}$ inside $\mathbb{R}^n$.
\end{definitionbox}
\subsection{Elementary Row Operations}
To solve a system of linear equations we associate an \textbf{augmented matrix} to the system of equations. For example:
$$
  \begin{array}
    {ccccccc}x & + & 2y & - & z  & = & 5 \\
    3x         & + & y  & - & 2z & = & 9 \\
    -x         & + & 4y & + & 2z & = & 0
  \end{array}
  \quad \Rightarrow \quad
  \begin{bmatrix}
    1  & 2 & -1 & | & 5 \\
    3  & 1 & -2 & | & 9 \\
    -1 & 4 & 2  & | & 0
  \end{bmatrix}
$$
To solve, we can perform the following \textbf{Elementary Row Operations (EROs)}:
\begin{enumerate}
  \item Multiply a row by a non-zero constant.
  \item Add a multiple of one row to another row.
  \item Swap two rows.
\end{enumerate}
The goal of these operations is to transform the augmented matrix into \textbf{row echelon form} (REF) or \textbf{reduced row echelon form} (RREF).

\subsubsection{REF and Strategy}
\begin{minipage}{0.8\textwidth}
  We say a matrix is in \textbf{row echelon form} (REF) if:
  \begin{itemize}
    \item The first non zero entry in each row is a 1 (called the \textbf{leading 1}).
    \item If a column has a leading 1, then all entries below it are 0.
    \item The leading 1 in each row is to the right of the leading 1 in the previous row.
    \item All rows of 0s are at the bottom of the matrix.
  \end{itemize}
\end{minipage}
\begin{minipage}{0.2\textwidth}
  \begin{center}
    $$
      \begin{bmatrix}
        1 & 2 & -1 & | & 3  \\
        0 & 1 & 2  & | & -1 \\
        0 & 0 & 1  & | & -1
      \end{bmatrix}
    $$
    \footnotesize \emph{Example of REF}
  \end{center}


\end{minipage} \\[2ex]
We have produced a new system of equations. This is easily solved by back substitution.

\begin{conceptbox}{Stategy for Obtaining REF}{}
  \begin{itemize}
    \item Get a 1 as the top left entry
    \item Use this 1 to clear the entries below it
    \item Move to the next column and repeat
    \item Continue until all leading 1s are in place
    \item Use back substitution to solve the system
  \end{itemize}
\end{conceptbox}
\subsubsection{Row Reduced Echelon Form}
\begin{minipage}{0.8\textwidth}
  A matrix is in \textbf{reduced row echelon form} (RREF) if:
  \begin{itemize}
    \item It is in REF
    \item The leading 1 in each row is the only non-zero entry in its column.
  \end{itemize}
\end{minipage}
\begin{minipage}{0.2\textwidth}
  \begin{center}
    $$
      \begin{bmatrix}
        1 & 0 & 0 & | & 2  \\
        0 & 1 & 0 & | & -1 \\
        0 & 0 & 1 & | & -1
      \end{bmatrix}
    $$
    \footnotesize \emph{Example of RREF}
  \end{center}
\end{minipage}

\section{Leading variables and free variables}
We'll start by an example:
$$
  \begin{array}{ccccccccc}
    x_1  & - & x_2  & - & x_3  & + & 2x_4 & = & 0 \\
    2x_1 & + & x_2  & - & x_3  & + & 2x_4 & = & 8 \\
    x_1  & - & 3x_2 & + & 2x_3 & + & 7x_4 & = & 2
  \end{array}
  \quad \Rightarrow \quad
  \begin{bmatrix}
    1 & -1 & -1 & 2 & | & 0 \\
    2 & 1  & -1 & 2 & | & 8 \\
    1 & -3 & 2  & 7 & | & 2
  \end{bmatrix}
$$
Solving this system of equations, we get:
$$
  \text{RREF:}\quad
  \begin{bmatrix}
    1 & 0 & 0 & 2  & | & 4 \\
    0 & 1 & 0 & -1 & | & 2 \\
    0 & 0 & 0 & 1  & | & 2
  \end{bmatrix}
$$




\end{document}
