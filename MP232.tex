\documentclass[12pt]{article}

%--------------------------------------------------------
% Basic Page Setup
%--------------------------------------------------------
\usepackage[margin=1in]{geometry} 
\usepackage{lmodern}
\usepackage[T1]{fontenc}
\usepackage[utf8]{inputenc}
\usepackage{microtype}  % Improves typography

% Slightly increase line spacing
\renewcommand{\baselinestretch}{1.2}

%--------------------------------------------------------
% Math Packages
%--------------------------------------------------------
\usepackage{amsmath,amssymb,amsthm}

%--------------------------------------------------------
% tcolorbox for Call-Out Blocks
%--------------------------------------------------------
\usepackage{tcolorbox}
\tcbuselibrary{theorems}

% Definition box
\newtcolorbox{definitionbox}{
  title=Definition,
  fonttitle=\bfseries,
  colback=blue!5!white,    % Light blue background
  colframe=blue!30!white,  % Darker blue frame
  coltitle=black,
  boxrule=1.5pt,
  arc=3pt,
  outer arc=3pt,
}

% Theorem box
\newtcolorbox{theorembox}{
  title=Theorem,
  fonttitle=\bfseries,
  colback=blue!5!black,
  colframe=blue!50!black,
  coltitle=black,
  boxrule=0.5pt,
  arc=4pt,
  outer arc=4pt,
}

% Note box
\newtcolorbox{notebox}{
  title=Note,
  fonttitle=\bfseries,
  colback=red!5!white,
  colframe=red!30!white,
  coltitle=black,
  boxrule=0.5pt,
  arc=4pt,
  outer arc=4pt,
}

%--------------------------------------------------------
% Theorem Environments (optional alternative)
%--------------------------------------------------------
\theoremstyle{definition}
\newtheorem{definition}{Definition}[section]
\newtheorem{example}[definition]{Example}

\theoremstyle{plain}
\newtheorem{theorem}[definition]{Theorem}


\title{
Mathematical Methods II\\[2ex]
Exams:\\
70\% Exam\\
30\% Continuous Assessment (3 parts)
}
\author{}     % Optional: Add author details if desired
\date{}       % Optional: Add date if desired
%--------------------------------------------------------
% Document
%--------------------------------------------------------
\begin{document}
\maketitle
\pagebreak

\tableofcontents
\pagebreak

\section{Week 1: Intro to Laplace Transforms}
\subsection{Preliminary: Exponential Functions}
Recall the following facts:
\begin{enumerate}
    \item \( e^t = \exp(t) = 1 + \frac{t^1}{1!} + \frac{t^2}{2!} + \frac{t^3}{3!} + \cdots 
           = \sum_{i=0}^\infty \frac{t^i}{i!}.\)
    \item \( e^0 = 1.\)
    \item As \( t \to \infty \), \( e^t \to \infty \);\quad as \( t \to -\infty \), \( e^t \to 0.\)
    \item \(\frac{d}{dt}\, e^t = e^t\),\quad and\quad \(\frac{d}{dt}\, e^{st} = s\, e^{st}.\)
    \item \(\displaystyle \int e^t\, dt = e^t + C,\quad \text{and} \quad
           \int e^{st}\, dt = \frac{1}{s}\, e^{st} + C.\)
    \item \( e^{t_1} \cdot e^{t_2} = e^{\,t_1 + t_2}.\)
\end{enumerate}


\subsection{Laplace Transforms}

\begin{definitionbox}
Consider a function \(f(t)\) for \(t > 0\).\\[1ex]
We define the Laplace transform of \(f(t)\) as
\[
\mathcal{L}\{f(t)\} = \int_0^\infty e^{-st} f(t) \, dt.
\]
\end{definitionbox}

\begin{notebox}
We can also write \( \mathcal{L}\{f(t)\} \) as \( F(s) \).\\[1ex]
Alternatively,
\[
\mathcal{L}\{f(t)\} = \lim_{R \to \infty} \int_0^R e^{-st} f(t) \, dt.
\]
\end{notebox}

\noindent Recalling that

$$\int_0^1 s t^2 \, dt = s\Big[\frac{t^3}{3}\Big]_0^1 = \frac{s}{3}$$
                     
\noindent we see that \( \mathcal{L}\{f(t)\} \) is just a function of \( s \).



\pagebreak
\end{document}
